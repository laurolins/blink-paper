\acknowledgements



\begin{center}{\it If I had the right to thank only one person, this
acknowledgement would be $\ldots$}
\end{center}

I would like to thank my supervisor S�stenes Lins, who is also my
father. I owe to him many things: to exist is obviously the most
important one, but here I want to mention the opportunity that
he gave me to, using my skills, contribute to the yet mysterious
field of 3-dimensional spaces. He presented me
an unexplored and elegant way to present spaces and said:
``Let's use computers to explore this''. And this is what we
did, and I am very happy with the process and with the result we
achieved.

\begin{center}
{\it but this constraint does not exist, so I can continue$\ldots$}
\end{center}

Thank you

Sofia, for filling my life with joy; my mother Bernardete and
my sister Isis, for your unrestricted support always; the new
generation: Pedrinho, Fernandinha, Jo�ozinhozinho, Arthur and Mariana; Nadja, Jorginho,
Jo�ozinho, Adelaide, Eneida, Heloiza, Dulce, Niedja,
vov� Lourdes, vov� Luiz, vov� Myriam and vov� Lauro for being a family
that makes me feel beloved; Roberta, Maria, and Fred for growing my family.

I am also grateful to: Silvio Melo, for his correct quantum
invariant implementation that guided me;
Paulo Soares, my best mate in the courses and
also my analysis tutor; Jalila and Donald Pianto, from whom I
learned in our joint works during our common disciplines;
Professors Klaus and Francisco Cribari from whom I learned
some probability and statistics; Val�ria, who kept me informed
of all bureaucracy and due dates; S�rgio Santa Cruz and Francisco
Brito, for their help with the hyperbolic plane; UFPE;
CAPES, for financial support. 