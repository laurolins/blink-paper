\abstract

A {\it blink} is a plane graph with its edges being red or green. A
{\it 3D-space} or, simply, a {\it space} is a connected, closed and oriented
3-manifold. In this work we explore in details, for the first time,
the fact that every blink induces a space and any space is induced by
some blink (actually infinitely many blinks). What is the space of a green triangle?
And of a red square? Are they the same? These questions were condensed into
a single one that guided a great part of the developed work: what are all
spaces induced by small blinks (few edges)? In this search we used a known
set of tools: the {\it blackboard framed links} (BFL),
the {\it homology groups}, the {\it quantum invariant} of Witten-Reshetikhin-Turaev,
the {\it 3-gems} and its simplification theory. Combining these tools with a
new theory of decomposition/composition of blinks we could identify all
spaces induced by blinks with up to 9 edges (or BFLs with up to 9 crossings).
Besides that, our effort resulted in an interactive computer program named
\textsc{Blink}. We hope that this program becomes useful in the study
of spaces, in particular, in the discovery of new invariants that
complement the quantum invariant and homology group solving the two
uncertainties that we left open in this work.

\vspace{1cm}

\par\vskip\baselineskip\noindent{{\bf Keywords}: 
topology, closed connected oriented 3-manifolds, plane graphs, 
spaces, graph encoded manifolds.
}