\resumo

Um {\it blink} � um grafo plano onde cada aresta ou � vermelha ou � verde. Um
{\it espa�o 3D} ou, simplesmente, um {\it espa�o} � uma variedade 3-dimensional
conexa, fechada e orientada. Neste trabalho exploramos pela primeira vez
em maiores detalhes o fato de que todo blink induz um espa�o e todo espa�o �
induzido por algum blink (na verdade por infinitos blinks). Qual o espa�o de
um tri�ngulo verde? E de um quadrado vermelho? S�o iguais? Estas perguntas
foram condensadas numa pergunta cuja busca pela resposta guiou em grande parte
o trabalho desenvolvido: quais s�o todos os espa�os induzidos por blinks
pequenos (poucas arestas)? Nesta busca lan�amos m�o de um conjunto
de ferramentas conhecidas: os {\it blackboard framed links} (BFL),
os {\it grupos de homologia}, o {\it invariante qu�ntico} de Witten-Reshetikhin-Turaev,
as {\it 3-gems} e sua teoria de simplifica��o. Combinamos a estas ferramentas
uma teoria nova de decomposi��o/composi��o de blinks e, com isso, conseguimos
identificar todos os espa�os induzidos por blinks de at� 9 arestas (ou BFLs
de at� 9 cruzamentos). Al�m disso, o nosso esfor�o resultou
tamb�m num programa interativo de computador chamado \textsc{Blink}.
Esperamos que ele se mostre �til no estudo de espa�os e, em particular,
na descoberta de novos invariantes que complementem o invariante qu�ntico
resolvendo as duas incertezas deixadas em aberto neste trabalho.

\vspace{1cm}

\par\vskip\baselineskip\noindent{{\bf Palavras-chave}:
topologia, 3-variedades fechadas conexas e orientadas, 
grafos planos, espa�os, {\it graph encoded manifolds}.
}